\documentclass[16pt]{article}

\usepackage{caption}

\usepackage{minted}

\usepackage{float}

\usepackage{hyperref}

\author{Farhad}
\title{It's {\LaTeX} Title}
\date{\today!}

\begin{document}

\maketitle

\newpage
\tableofcontents
\newpage

\begin{abstract}
 \LaTeX \hspace{1pt} documentation written as \LaTeX! How novel and totally not
 my idea!
\end{abstract}

\section*{Introduction}         %
Some Text is for intro. A reference to sublist \ref{blah}

\subsection{sub intro}

Some normal text for sub intro.

\textbf{some bold text here}.

\textit{italic stuff}.

\emph{this is emphatic}

\underline{some underline text}

``this is in double quote''

`this is in single quote'

\section{Lists}
\begin{enumerate}
\item foo
\item bar
\item zoo
\end{enumerate}

\subsection{sublist \label{blah}}
\begin{itemize}
\item foo
  \begin{itemize}
  \item one
  \item twoo
  \item three
  \end{itemize}
\item bar
\item zoo
\end{itemize}

\section{Math}

Sets and relations play a vital role in many mathematical research papers.
Here's how you state all x that belong to X, $\forall$ x $\in$ X. \\

\[a^2 + b^2 = c^2 \]

My favorite Greek letter is $\xi$. I also like $\beta$, $\gamma$ and $\sigma$.
I haven't found a Greek letter yet that \LaTeX \hspace{1pt} doesn't know
about! \\


Operators are essential parts of a mathematical document:
trigonometric functions ($\sin$, $\cos$, $\tan$),
logarithms and exponentials ($\log$, $\exp$),
limits ($\lim$), etc.
have per-defined LaTeX commands.
Let's write an equation to see how it's done:
$\cos(2\theta) = \cos^{2}(\theta) - \sin^{2}(\theta)$ \\

$$ ^{10}/_{7} $$

$$ \frac{n!}{k!(n - k)!} $$ \\


% Display math with the equation 'environment'
\begin{equation} % enters math-mode
  c^2 = a^2 + b^2.
  \label{eq:pythagoras} % for referencing
\end{equation} % all \begin statements must have an end statement

\begin{equation}
  \sum_{i=0}^{5} f_{i}
\end{equation}

\begin{equation}
  \int_{0}^{\infty} \mathrm{e}^{-x} \mathrm{d}x
\end{equation}

\begin{figure}[H] % H here denoted the placement option.
    \centering % centers the figure on the page
    % Inserts a figure scaled to 0.8 the width of the page.
    %\includegraphics[width=0.8\linewidth]{right-triangle.png}
    % Commented out for compilation purposes. Please use your imagination.
    \caption{reset with sides $a$, $b$, $c$}
    \label{fig:reset}
\end{figure}


\begin{table}[H]
  \caption{Caption for the Table.}
  % the {} arguments below describe how each row of the table is drawn.
  % Again, I have to look these up. Each. And. Every. Time.
  \begin{tabular}{c|cc}
    Number &  Last Name & First Name \\ % Column rows are separated by &
    \hline % a horizontal line
    1 & Biggus & Dickus \\
    2 & Monty & Python
  \end{tabular}
\end{table}

\begin{verbatim}
  print("Hello World!")
  a%b; % look! We can use % signs in verbatim.
  random = 4; #decided by fair random dice roll
\end{verbatim}

\begin{minted}{python}
import numpy as np

def incmatrix(genl1,genl2):
    m = len(genl1)
    n = len(genl2)
    M = None #to become the incidence matrix
    VT = np.zeros((n*m,1), int)  #dummy variable

    #compute the bitwise xor matrix
    M1 = bitxormatrix(genl1)
    M2 = np.triu(bitxormatrix(genl2),1)

    for i in range(m-1):
        for j in range(i+1, m):
            [r,c] = np.where(M2 == M1[i,j])
            for k in range(len(r)):
                VT[(i)*n + r[k]] = 1;
                VT[(i)*n + c[k]] = 1;
                VT[(j)*n + r[k]] = 1;
                VT[(j)*n + c[k]] = 1;

                if M is None:
                    M = np.copy(VT)
                else:
                    M = np.concatenate((M, VT), 1)

                VT = np.zeros((n*m,1), int)

    return M
\end{minted}

\begin{minted}{javaScript}
  console.log("foo");
\end{minted}


\url{https://learnxinyminutes.com/docs/latex/}, or
\href{https://learnxinyminutes.com/docs/latex/}{shadowed by text}



\end{document}