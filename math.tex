\documentclass{article}
\usepackage{amssymb}
\author{Farhad}
\title{my precalculus notes}
\date{\today}

\begin{document}
\maketitle

\section*{Fundamentals}

Natural numbers: $ 1, 2, 3, ... $ \\
Integers: $..., -2, -1, 0, 1, 2, ...$ \\
Rational numbers: $ r = \frac{m}{n} $ where m and n are integers and n != 0 \\
Irrational numbers: $ \sqrt{2}, \sqrt{5}, \sqrt[3]{2}, \pi  $ \\
\newline
Commutative Property: $ ab = ba $

Associative Properties: $ (ab)c = a(bc) $

Distributive Property: $ a(b + c) = ab + ac $

$0$ is additive identity: $ a + 0 = a $ for any real number a

$1$ is multiplicative identity: $ a . 1 = a $ for any real numbr a

Every nonzero real number a has an inverse: $ a . (\frac{1}{a}) = 1 $

We refer to $\frac{a}{b}$ as the \emph{quotient} of a and b or as the fraction a over b; a is the \emph{numerator} and b is \emph{denominator} (or divisor).

\subsection*{Properties of Fractions}
\begin{itemize}
\item $ \frac{a}{b} \div \frac{c}{d} = \frac{a}{b} . \frac{d}{c} $ When dividing fractions, invert the divisor and multiply.
\item If $ \frac{a}{b} = \frac{c}{d} $, then $ ad = bc $: Cross-multiply.
\end{itemize}

\textbf{Least Common Denominator (LCD)}: \\
We find the least common denominator (LCD) by forming the product of
all the prime factors that occur in these factorizations, using the
highest power of each prime factor. \textbf{Use division ladder technique}

\subsection*{Sets and Intervals}
A set is a collection of objects, and these objects are called the elements of the set.
$a \in S$ means that a is an element of S and $b \not\in S$ means that b is not an element of S.
\newline
\textbf{set-builder notation}:
$$A = \{\,x \mid x \mbox{ is an integer and } 0 < x < 7 \,\}$$
Which is read ``A is the set of all x such that is an integer and $0 < x < 7$ ''
\newline
If \emph{S} and \emph{T} are sets, then their \textbf{union} $S \cup T$ is the set that consists of all elements that are in \emph{S} \textit{or} \emph{T} (or in both). The \textbf{intersection} of \emph{S} and \emph{T} is the set $S \cap T$ consisting of all elements that are in both \emph{S} \textit{and} \emph{T}. In other words, $S \cap T$ is the common part of \emph{S} and \emph{T}. The \textbf{empty set}, denoted by $\emptyset$, isthe set that contains no element.

\end{document}
